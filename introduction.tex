
\begin{savequote}[.5\linewidth]
  ``Growth for the sake of growth is the ideology of the cancer cell.''
  \qauthor{- Edward Abbey}
\end{savequote}
\chapter{Introduction}
Cancer has been with human species throughout our 4000 years of history. Although our understanding of cancer has changed drastically through time, but its treatment stays challenging and in many cases we cannot cure it to this date. The immense frustration of dealing with cancer not only affects the patients, but also the doctors, pathologists, and oncologists treating those patients. Siddhartha Mukherjee explains the feeling with these words~\cite[prologue]{the-emperor-of-all-maladies}:

\begin{displayquote}
  ...
  
  There were seven such cancer fellows at this hospital. On paper, we seemed like a formidable force: graduates of five medical schools and four teaching hospitals, sixty-six years of medical and scientific training, and twelve postgraduate degrees among us. But none of those years or degrees could possibly have prepared us for this training program. Medical school, internship, and residency had been physically and emotionally grueling, but the first months of the fellowship flicked away those memories as if all of that had been child's play, the kindergarten of medical training.

  ...

  The stories of my patients consumed me, and the decisions that I made haunted me. \emph{Was it worthwhile continuing yet another round of chemotherapy on a sixty-six-year-old pharmacist with lung cancer who had failed all other drugs? Was it better to try a tested and potent combination of drugs on a twenty-six-year-old woman with Hodgkin's disease and risk losing her fertility, or to choose a more experimental combination that might spare it? Should a Spanish-speaking mother of three with colon cancer be enrolled in a new clinical trial when she can barely read the formal and inscrutable language of the consent forms?}
\end{displayquote}

These decisions are hard to make, and it does not help knowing that our deep understanding of cancer is far from complete. The battle against cancer has many fronts, including prevention, diagnosis, and treatment, all of which benefiting from the advancements in understanding the disease. Cancer researchers try to understand the disease in the lab, and once their findings are confirmed and accepted by the community, they are used by pathologists and oncologists in the clinics. However, the diagnosis itself is also complex, challenging, and uncertain. This is why in many cases doctors do not agree on the exact diagnosis, and a counsel of doctors is required for a better and more reliable diagnosis.

I can better put this thesis in context by giving my personal perspective on cancer research and diagnosis. I spent over a year researching in British Columbia's Cancer Research Center in Vancouver, Canada, which also admitted patients for diagnosis and treatment. As a result, I worked closely with oncologists diagnosing patients as well as  cell biologists researching fundamentals of cancer. When it came to diagnosis, I was motivated by questions such as ones described bellow.

Some patients enter the clinic carrying cancer type A, which is mild and does not require an aggressive treatment. Therefore they are put under the appropriate treatment while their condition is monitored through time. The disease in some of these patients develops into another type, let say type B, which is more aggressive and sometimes requires a harsher or a different treatment. Like many other diseases, cancer can be defeated best in its earliest stages. This is why we could potentially achieve better prognosis for these patients had we known their disease will develop to type B, before it happens.

Similar to the above issue, out of the many patients who go in remission, some relapse with a cancer which is significantly more resistant to usual treatments compared to when they where originally diagnosed. This is sometimes due to a very small portion of the original cancer being or becoming resistance to the treatment. The challenge is that this portion of the original disease is so small, often in the order of a few cells, that it goes undetected at the time of diagnosis or during the treatment, and it emerges months after the original treatment is over. Now the question is, looking back at the data of these patients, could we detect those cells, or something about the original cancer cells predicting the relapse, earlier during the treatment or even at the time of diagnosis?

In that cancer research center, people were also researching cancer by looking at the effect of different drugs and drug combinations targeting different genes. Some of cell biologist friends would choose a few genes, often taking recommendations from their supervisors, and spend years investigating the role of those genes in the development of a particular cancer type. Of course they would do their best to choose the most relevant set of genes to their knowledge, but the task of choosing a few genes out of over 22k genes in the humans is rather challenging and does not always lead to successful treatments. If we had computational methods which would give us a list of promising genes that are influential in determining the cancer subtypes, the same set of genes could be a better starting point for cell biologists to choose from. Doing so, if we could deliver a list of genes specific to each patient, a rather personalized treatment could be selected for the patient.

TODO: bridge medicine and machine learning

This thesis is an effort to give cancer researchers better clues and help them in their work, and also help oncologists and clinicians better diagnose the patients. Chapter~\ref{sec:background} covers some of the basics required to follow the later chapters. In chapter~\ref{sec:fcs} we focus on the clinical diagnosis and analyze the kind of data used on a daily basis in clinics. Then we continue in chapter~\ref{sec:adaptive-learning} by focusing on a type of data used in cancer research laps, introducing methods for a personalized diagnosis for the patients.

