
\begin{savequote}[.5\linewidth]
  ``Growth for the sake of growth is the ideology of the cancer cell.''
  \qauthor{- Edward Abbey}
\end{savequote}
\chapter{Introduction}
\label{ch:intro}
Cancer has been with human species throughout our 4000 years of history.
Although our understanding of cancer has changed drastically through time, its
treatment stays challenging and in many cases we have not been able to cure it
to this date. The immense frustration of dealing with cancer not only affects
the patients, but also the doctors, pathologists, and oncologists treating
those patients. Siddhartha Mukherjee in his book ``The Emperor of All
Maladies'' explains the feeling with these
words~\cite[prologue]{the-emperor-of-all-maladies}:

\begin{displayquote}
  ...
  
  There were seven such cancer fellows at this hospital. On paper, we seemed
  like a formidable force: graduates of five medical schools and four teaching
  hospitals, sixty-six years of medical and scientific training, and twelve
  postgraduate degrees among us. But none of those years or degrees could
  possibly have prepared us for this training program. Medical school,
  internship, and residency had been physically and emotionally grueling, but
  the first months of the fellowship flicked away those memories as if all of
  that had been child's play, the kindergarten of medical training.

  ...

  The stories of my patients consumed me, and the decisions that I made haunted
  me. \emph{Was it worthwhile continuing yet another round of chemotherapy on a
    sixty-six-year-old pharmacist with lung cancer who had failed all other
    drugs? Was it better to try a tested and potent combination of drugs on a
    twenty-six-year-old woman with Hodgkin's disease and risk losing her
    fertility, or to choose a more experimental combination that might spare
    it? Should a Spanish-speaking mother of three with colon cancer be enrolled
    in a new clinical trial when she can barely read the formal and inscrutable
    language of the consent forms?}
\end{displayquote}

I can better put this thesis in context by giving my personal perspective on
cancer research and diagnosis. I spent over a year researching in British
Columbia's Cancer Research Center (BCCRC) in Vancouver, Canada, which also
admitted patients for diagnosis and treatment. As a result, I worked closely
with oncologists who were diagnosing patients as well as cell biologists
researching fundamentals of cancer and new treatments for it. Their experiences
and the challenges they were facing had a great impact on me and to a large
extent shaped my research for the next few following years. Then, once I moved
to the Bioinformatics lab in Max Planck Institute for Informatics in
Saarbr\"ucken, Germany, I was better equipped with required the statistics and
machine learning skills to tackle the computational problems explained here.
Here is a list of some of the challenges I saw people were facing in BCCRC:

\begin{itemize}
  \item Some patients enter the clinic carrying cancer type A, which is mild
    and does not require an aggressive treatment. Therefore they are put under
    the appropriate treatment while their condition is monitored through time.
    However, the disease in some of these patients develops into another type,
    let say type B, which is more aggressive and sometimes requires a harsher
    or a different treatment. Considering the fact that like many other
    diseases, cancer can be defeated best while at its earliest stages, we
    could potentially achieve a better prognosis for these patients had we
    known their disease will develop into type B at a much earlier stage.

  \item Similar to the above issue, out of the many patients who go in
    remission, \emph{i.e} they seem free of cancer after the course of the
    treatment, some relapse with a cancer which is significantly more resistant
    to usual treatments compared to when they where originally diagnosed. This
    sometimes happens when a very small number of cells from the original
    cancer are or become resistant to the drugs and survive the treatment, but
    go undetected for a while in the tests and scans. It may take months for
    those cell populations to grow large enough to be detected again. Now the
    question is, looking back at the data of these patients, could we detect
    those cells, or something about the original cancer cells predicting the
    relapse, earlier during the treatment or even at the time of the original
    diagnosis?

  \item In BCCRC, people were also researching cancer by looking at the effects
    of different drugs and drug combinations targeting different genes. Some of
    my cell biologist friends, often taking recommendations from their
    supervisors, would choose a few genes and spend years investigating the
    role of those genes in the development of a particular type of cancer. Of
    course they would do their best to choose the most relevant set of genes,
    but the task of choosing a few genes out of over 22k genes on the human
    genome is rather challenging and does not always lead to successful
    treatments and positive results. There is also a bias towards the genes
    which have been discovered earlier and have been studied more in depth. If
    the cancer happens to be related to one of the less studied genes, it
    usually can stay under the radar for while.
\end{itemize}

All of the abovementioned issues involve decisions that are hard to make, and
it does not help knowing that our deep understanding of cancer is far from
complete. The battle against cancer has many fronts, including prevention,
diagnosis, and treatment, all of which benefiting from the advancements in
understanding the disease. As a part of the process, cancer researchers try to
understand the disease in the lab, and once their findings are confirmed,
accepted by the community, and pass the legal requirements, they are used by
pathologists and oncologists in the clinics. However, the diagnosis itself is
also complex, challenging, and in many cases not a definitive one. This is why
sometimes doctors do not agree on the exact diagnosis, and a counsel of experts
is required for a better and more reliable diagnosis and a treatment which
hopefully results in a better prognosis.

To better understand the challenge, we need to realize that cancer is a
collection of extremely smart and complicated diseases. Although they share
many common characteristics, the same treatment does not result in a similar
prognosis. For example, two patients may come with two physically very similar
malignant tumors in their breasts. However, one of the patient's tumor grows in
response to estrogen, while the other one shows no reaction to estrogen. In
this case, a treatment which blocks estrogen receptors is very effective for
the first patient (ER+), while being completely ineffective against the second
patient who has an ER- subtype of breast cancer.

At the core of it, it comes to the fact that in normal cells there are
processes and checks put in place which define when and if the cell should
divide or die at a certain time or under certain conditions. Some of those
mechanisms act like an automated self destruct switch which is triggered if
something goes wrong in the cell. However, our cells are under constant stress
from the external factors which damage them, UV being one example, and
sometimes the damage to the cell affects those mentioned mechanisms and
disables them. This may lead the cell to divide uncontrollably, and become
cancerous. Another difference between normal cells and cancerous ones, is that
normal cells are capable of repairing most of the mutations happening on their
DNA as a result of either external stress or during cell division, whereas
those processes themselves are often damaged in a cancerous cell. As a result,
the rate of mutation in cancerous cells is higher of a few orders of magnitude
compared to a normal healthy cell. The high rate of cell division in
combination with hyper mutation, makes cancerous cells very adaptive to their
environment, as well as against the drugs attacking them.

Cancer treatments are methods and substances which ideally target only the
cancerous cells and kill them, or stop their growth and cell division. In a
sense, they are poison, but ideally only to cancer. However, cancer cells are
derived from our own cells, and therefore it is not easy to distinguish them
from normal cells. The more difference between the cancer cells and our normal
healthy cells, the easier it is to target them; but unfortunately not all
cancer subtypes are easily distinguishable from healthy cells for the purpose
of treatment.

In the first part of this thesis, \emph{i.e.} Chapter~\ref{sec:fcs}, we tackle
some of the issues faced by clinicians while analyzing flow cytometry data. A
flow cytometer, \emph{i.e.} the instrument producing the data, is capable of
measuring a dozen or so different characteristics of individual cells in a
given sample, which in our case are all from biopsies or blood samples taken
from patients. This type of single cell measurement data enables us to detect
and sort different cell populations within a single given sample.
Chapter~\ref{sec:fcs} explains our designed and implemented methods to analyze
flow cytometry to address the following challenges:

\begin{itemize}
  \item Automated analysis of the data: manual analysis of flow cytometry data
    involves manually looking at 2D projections of the data long two selected
    features at a time, and potentially filtering a part of the data before
    moving on to a different projection. This way at each step a
    sub-population, \emph{i.e.} a subset of the cells, is selected for further
    analysis. Presence or absence of a cell population, or its predominance
    compared to other cell populations, can be indicative of a certain type of
    cancer. Our developed methods can automate this process and help clinicians
    to quickly find a certain cell population within a given sample.
  \item Novel cell population discovery: our method analyzes the relationship
    between all the cell populations it finds with the target cancer subtypes,
    and reports the cell populations which seem to be informative or predictive
    in differentiating two subtypes. Discovering new predictive cell
    populations can help the diagnosis process, as well as potentially a better
    choice of treatment for each patient.
  \item Quality assurance: in clinics, oncologists sometimes go through a QA
    process in which they randomly look at a limited number of past patients
    and retrospectively check the quality of the original diagnosis. However,
    due to limited amount of available resources, the can only go through a
    small number of cases. We develop a method to report cases where there is a
    higher probability that the diagnosis could be improved, hoping that
    retrospective investigation of those cases could generally improve the
    diagnosis process.
  \item Visualization: one important aspect of analyzing cell populations is
    that they are related to one another, \emph{i.e.} some cell populations are
    a subset of others. While there is value in reporting relevant cell
    populations for a diagnosis, it is also essential to visualize related
    sub-populations to the reported ones, as well as showing how one can best
    filter and find a specific group of cells if they were to do that manually.
    Another reason to visualize the data the way we do, is that it helps a
    cancer researcher to better understand why or how those cell populations
    are relevant.
  \item Enabling similar tests using cheaper machines: one important benefit of
    visualizing the cell populations the way we do, is that it shows
    alternative ways that a cell population can be isolated from the other
    cells using fewer measurements per cell. As a result, the same cell
    population can then be filtered using a smaller number or a cheaper
    variation of chemical reagents used in flow cytometry. Consequently, a
    pathologist in a place where there is harder or no access to more expensive
    cytometers and reagents, can benefit from the findings of a research
    institute with significantly more budget at their disposal.
\end{itemize}

The second part of the thesis, described in
Chapter~\ref{sec:adaptive-learning}, focuses on some types of data that are not
yet available as a routine test in clinics, but are essential to our deeper
understanding of cancer. These data are measurements from the whole human
genome, and our focus is on gene expression profiles and a modification on the
DNA, called methylation, \emph{i.e.} a methyl component is attached a
Cytosine(C) or an Adenine(A) base on the DNA. A gene expression profile
measures the activity of all 22k+ genes in a given sample, and a usual DNA
methylation profile measures the methylation level of about 450k sites on the
human DNA. DNA methylation levels can change due to environmental factors and
during cell differentiation and aging; there is also evidence that they are
sometimes heritable. These data, as well as others such as DNA, RNA, and
protein sequence data have been increasingly used by biologists and
computational biologists to better understand cell biology in many fields,
including cancer research. These data and methods have been so essential to our
understanding of cancer that the classification of some cancer types now depend
on them. In some cases, molecular and chemical analysis of cancer has shown us
that two different classes of cancer are indeed the same disease, only in
different stages. Lymphoma and leukemia are two good examples which used to be
considered two different cancers, and now together they form the \emph{lymphoid
  neoplasms} group, since on the molecular level they have a lot in common.

Although there has been magnificent advancements in the field from the
computational perspective, the computational problems are still considered very
hard problems. The curse of dimensionality on top of the low number of samples
compared to the number of dimensions in the data all result in a hard
computational problem as explained in detail in Chapter~\ref{sec:background}).
Also, the fact that the data is often affected by noise and batch effects
doesn't help the case neither, which we cover in more detail in
Chapter~\ref{sec:adaptive-learning}. Another challenging factor is that cancer,
and even a single cancer tumor, is heterogeneous and different cases show very
different genetic profiles. Historically this has lead to further
classification of cancer, and at times, changing the classification and merging
some classes all together as mentioned above. As a result, ideally when a
patient is prescribed a treatment, that treatment is fine tuned and adapted as
much as possible to tackle the patient's specific cancer. This fine tuning,
which covers the spectrum from choosing the best combination of drugs to
designing and manufacturing a drug specific to the patient and the patient
alone, is called personalized medicine. With a focus on adaptive ad
interpretable models, Chapter~\ref{sec:adaptive-learning} addresses the
following challenges:

\begin{itemize}
  \item Per-patient significant gene/genome region discovery: our developed
    models are adaptive, \emph{i.e.} for any given data from a single patient,
    the model can give some information about the potential underlying cause of
    the disease for the specific patient. The provided information would give
    clues to practitioners to better find a treatment for the patient.
  \item Per-disease significant gene/genome region discovery: in order to help
    cancer researchers studying cancer and cancer treatments, we design
    computational methods capable of reporting a list of promising genes that
    are influential in determining the cancer subtypes, so that the same set of
    genes can be a better starting point for cell biologists to choose and
    investigate.
  \item Interpretability, per-patient, per-disease: our models investigate
    genes in networks and take their relationships between one another which
    can be seen from the data into account. Interpreting and visualizing the
    models to show the significance of each gene as well as their relationship
    helps pathologists and cancer cell biologists to better trust the methods
    and understand how the model.
\end{itemize}

In the following chapters, some of the basics required to follow the later
sections, including some related concepts in machine learning, graph theory,
and biology are covered in Chapter~\ref{sec:background}. In
Chapter~\ref{sec:fcs} we focus on flow cytometry data and explain the design
and implementation of a few pieces which together make an end to end pipeline
to analyze such data, and apply that to some specific lymphoma subtypes. Then
we continue in Chapter~\ref{sec:adaptive-learning} with the analysis of mostly
DNA methylation data and design some adaptable and interpretable models with an
eye on personalized medicine.
