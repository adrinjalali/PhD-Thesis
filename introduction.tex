
\begin{savequote}[.5\linewidth]
  ``Growth for the sake of growth is the ideology of the cancer cell.''
  \qauthor{- Edward Abbey}
\end{savequote}
\chapter{Introduction}
\label{ch:intro}
Cancer has been with human species throughout our 4000 years of history.
Although our understanding of cancer has changed drastically through time, its
treatment stays challenging and in many cases we have not been able to cure it
to this date. The immense frustration of dealing with cancer not only affects
the patients, but also the doctors, pathologists, and oncologists treating those
patients. Siddhartha Mukherjee in his book ``The Emperor of All Maladies''
explains the feeling with these
words~\cite[prologue]{the-emperor-of-all-maladies}:

\begin{displayquote}
  ...
  
  There were seven such cancer fellows at this hospital. On paper, we seemed
  like a formidable force: graduates of five medical schools and four teaching
  hospitals, sixty-six years of medical and scientific training, and twelve
  postgraduate degrees among us. But none of those years or degrees could
  possibly have prepared us for this training program. Medical school,
  internship, and residency had been physically and emotionally grueling, but
  the first months of the fellowship flicked away those memories as if all of
  that had been child's play, the kindergarten of medical training.

  ...

  The stories of my patients consumed me, and the decisions that I made haunted
  me. \emph{Was it worthwhile continuing yet another round of chemotherapy on a
    sixty-six-year-old pharmacist with lung cancer who had failed all other
    drugs? Was it better to try a tested and potent combination of drugs on a
    twenty-six-year-old woman with Hodgkin's disease and risk losing her
    fertility, or to choose a more experimental combination that might spare it?
    Should a Spanish-speaking mother of three with colon cancer be enrolled in a
    new clinical trial when she can barely read the formal and inscrutable
    language of the consent forms?}
\end{displayquote}

These decisions are hard to make, and it does not help knowing that our deep
understanding of cancer is far from complete. The battle against cancer has many
fronts, including prevention, diagnosis, and treatment, all of which benefiting
from the advancements in understanding the disease. Cancer researchers try to
understand the disease in the lab, and once their findings are confirmed and
accepted by the community, they are used by pathologists and oncologists in the
clinics. However, the diagnosis itself is also complex, challenging, and
uncertain. This is why in many cases doctors do not agree on the exact
diagnosis, and a counsel of experts is required for a better and more reliable
diagnosis.

Cancer is a collection of extremely smart and complicated diseases. Although
they share many common characteristics, they cannot be treated the same. A
treatment which is very effective for some subtype of breast cancer, may be
completely ineffective against another subtype while both showing very similar
symptoms. In normal cells, there are processes and checks put in place which
define when and if the cell should divide, or die at a certain time or under
certain conditions. Some of those mechanisms act like an automated self destruct
switch which is triggered if something goes wrong in the cell. However,
sometimes the damage to the cell affects those mechanisms and disables them.
This may lead to the cell divide uncontrollably, and become cancerous. Another
difference between normal cells and cancerous ones, is that normal cells are
capable of repairing most of the mutations happening on the DNA, whereas those
processes themselves are damaged in a cancerous cell. As a result, the rate of
mutation in cancerous cells is higher of a few orders of magnitude. The high
rate of cell division in combination with hyper mutation, makes cancerous cells
very adaptive to the environment around them, as well as against the drugs
attacking them.

Cancer cells are derived from our own cells, and therefore it is not easy to
distinguish them from normal cells.

I can better put this thesis in context by giving my personal perspective on
cancer research and diagnosis. I spent over a year researching in British
Columbia's Cancer Research Center (BCCRC) in Vancouver, Canada, which also
admitted patients for diagnosis and treatment. As a result, I worked closely
with oncologists diagnosing patients as well as cell biologists researching
fundamentals of cancer. Their experiences and the challenges they were facing
had a great impact on me and to a large extent shaped my research for the next
few following years. More specifically, here are some of the questions which
intrigued me the most.

Some patients enter the clinic carrying cancer type A, which is mild and does
not require an aggressive treatment. Therefore they are put under the
appropriate treatment while their condition is monitored through time. However,
the disease in some of these patients develops into another type, let say type
B, which is more aggressive and sometimes requires a harsher or a different
treatment. Considering the fact that like many other diseases, cancer can be
defeated best in its earliest stages, we could potentially achieve a
better prognosis for these patients had we known their disease will develop into
type B at a much earlier stage.

Similar to the above issue, out of the many patients who go in remission,
\emph{i.e} they seem free of cancer after the course of the treatment, some
relapse with a cancer which is significantly more resistant to usual treatments
compared to when they where originally diagnosed. This sometimes happens when a
very small number of cells from the original cancer are or become resistant to
the drugs and survive the treatment, but go undetected for a while in the tests
and scans. It may take months for them to grow large enough to be detected
again. Now the question is, looking back at the data of these patients, could we
detect those cells, or something about the original cancer cells predicting the
relapse, earlier during the treatment or even at the time of the original
diagnosis?

In BCCRC, people were also researching cancer by looking at the effect of
different drugs and drug combinations targeting different genes. Some of my cell
biologist friends, often taking recommendations from their supervisors, would
choose a few genes and spend years investigating the role of those genes in the
development of a particular cancer type. Of course they would do their best to
choose the most relevant set of genes to their knowledge, but the task of
choosing a few genes out of over 22k genes on the human genome is rather
challenging and does not always lead to positive results and successful
treatments.

This thesis is an effort to design and develop computational methods to tackle
the abovementioned challenges. There are two main parts to this thesis. In the
first part, presented in Chapter~\ref{sec:fcs}, the focus is mostly on the
methods applicable directly in clinics, using the kind of data readily available
at the time of the diagnosis. The technology required to produce the required
data is not new and it has been in widespread use for decades all around the
world. The technology is flow cytometry and the machine producing the data is a
flow cytometer. It takes certain measurements from single cells, one by one,
allowing us to study the presence and absence of certain cell types in a single
biopsy. Although the technology has been around for a while, up until recently
researchers have been analyzing its data manually. This work presents methods to
automate some of those laborious and tedious tasks, while deriving new
information and insights at the same time. As a byproduct, these methods can be
used to design tests which can be run on cheaper machines available in poorer
countries and yet deliver a relatively similar performance.

More specifically, one way pathologists study a given biopsy, is to find out
what kinds of cells are present in a given sample, and what types of
irregularities they posses. Each of these groups of cells is called a cell
population, and what is presented in Chapter~\ref{sec:fcs}, is an effort to find
different ways of discovering meaningful cell populations in each given sample.
At the same time, we detect certain cell populations which appear to be
correlated with a specific disease.

The other part of this thesis is closer to the bleeding edge of cancer research
than what happens in clinics. It uses some types of data, such as gene
expression profiles and DNA methylation data, which are more expensive to
acquire and are usually measured for the purpose of research. A gene expression
profile measures the activity of all 22k+ genes for a given sample, and a usual
DNA methylation profile measures the methylation level of about 450k sites on
the human DNA. These data, as well as others such as DNA, RNA, and protein
sequence data have been increasingly used by biologists and computational
biologists to better understand cell biology in many fields, including cancer
research.

These methods have been so essential to our understanding of cancer, that the
classification of some cancer types now depend on them. In some cases, they have
shown us that two different classes of cancer are indeed the same disease, only
in different stages.

If we had computational methods which would give us a list of promising genes
that are influential in determining the cancer subtypes, the same set of genes
could be a better starting point for cell biologists to choose from. Doing so,
if we could deliver a list of genes specific to each patient, a rather
personalized treatment could be selected for the patient.

Although there has been magnificent advancements in the field, from the
computational perspective, these are very hard problems. The curse of
dimensionality on top of low number of samples compared to dimension count makes
it really hard. The fact that the data is often affected by noise and batch
effects (Section ???) doesn't help the case neither. another category of
challenges include the facts that cancer, and even a single cancer tumor, is
heterogeneous and different cases show very different varying genetic profiles.
Historically, this has lead to further classification of cancer, ant at times,
changing the classification and merging some classes all together.

Black box models are not good. Interpreting models has value, and we do that.
Medicine has also progressed. We have many treatments for the same diagnosis. We
would like to help the system towards personalized medicine.

This thesis is an effort to give cancer researchers better clues and help them
in their work, and also help oncologists and clinicians better diagnose the
patients. Chapter~\ref{sec:background} covers some of the basics required to
follow the later chapters. In chapter~\ref{sec:fcs} we focus on the clinical
diagnosis and analyze the kind of data used on a daily basis in clinics. Then we
continue in chapter~\ref{sec:adaptive-learning} by focusing on a type of data
used in cancer research laps, introducing methods for a personalized diagnosis
for the patients.
