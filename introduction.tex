
\begin{savequote}[.5\linewidth]
  ``Growth for the sake of growth is the ideology of the cancer cell.''
  \qauthor{- Edward Abbey}
\end{savequote}
\chapter{Introduction}
Cancer has been with human species throughout our 4000 years of history. Although our understanding of cancer has changed drastically through time, but its treatment stays challenging and in many cases we cannot cure it to this date. The immense frustration of dealing with cancer not only affects the patients, but also the doctors, pathologists, and oncologists treating those patients. Siddhartha Mukherjee explains the feeling with these words~\cite[prologue]{the-emperor-of-all-maladies}:

\begin{displayquote}
  ...
  
  There were seven such cancer fellows at this hospital. On paper, we seemed like a formidable force: graduates of five medical schools and four teaching hospitals, sixty-six years of medical and scientific training, and twelve postgraduate degrees among us. But none of those years or degrees could possibly have prepared us for this training program. Medical school, internship, and residency had been physically and emotionally grueling, but the first months of the fellowship flicked away those memories as if all of that had been child's play, the kindergarten of medical training.

  ...

  The stories of my patients consumed me, and the decisions that I made haunted me. \emph{Was it worthwhile continuing yet another round of chemotherapy on a sixty-six-year-old pharmacist with lung cancer who had failed all other drugs? Was it better to try a tested and potent combination of drugs on a twenty-six-year-old woman with Hodgkin's disease and risk losing her fertility, or to choose a more experimental combination that might spare it? Should a Spanish-speaking mother of three with colon cancer be enrolled in a new clinical trial when she can barely read the formal and inscrutable language of the consent forms?}
\end{displayquote}

These decisions are hard to make, and it does not help knowing that our deep understanding of cancer is far from complete. The battle against cancer has many fronts, including prevention, diagnosis, and treatment, all of which benefiting from the advancements in understanding the disease. Cancer researchers try to understand the disease in the lab, and once their findings are confirmed and accepted by the community, they are used by pathologists and oncologists in the clinics. However, the diagnosis itself is also complex, challenging, and uncertain. This is why in many cases doctors do not agree on the exact diagnosis, and a counsel of doctors is required for a better and more reliable diagnosis.

This work (TODO: thesis?) is an effort to give cancer researchers better clues and help them in their work, and also help oncologists and clinicians better diagnose the patients.

Chapter~\ref{sec:background} covers some of the basics required to follow the later chapters. In chapter~\ref{sec:fcs} we focus on the clinical diagnosis and analyze the kind of data used on a daily basis in clinics. Then we continue in chapter~\ref{sec:adaptive-learning} by focusing on a type of data used in cancer research laps, introducing methods for a personalized diagnosis for the patients.

