\chapter{Flow Cytometry Analysis}

Here we talk about flow cytometry data, how we analyze and visualize it; and how we use that analysis alongside with some machine learning tools to classify samples into cancer subtypes.

\section{Flow Cytometry}
Flow cytometry is a technology that allows measurement of biomarkers inside and outside cells on a single cell basis~\cite{flow-cytometry}. The technology can also sort and separate certain cells according to a given criterion~\cite{flow-cytometry-sorting}.

Cell preparation in flow cytometry involves suspension of the cells in a liquid containing biomarker reagents. Reagents are marked antibodies that can be detected by the laser beams in the flow cytometer machine~\cite{flow-cytometry-cell-preparation}. The antibodies are usually marked with a fluorescent label. Each fluorescent marker has a corresponding peak excitation and emission wavelength which can be detected using lasers or lamps available on the flow cytometer machine. The combination of markers has to be chosen such that their corresponding wavelengths have minimal overlap; otherwise they cannot be distinguished from one another due to interference between them.

In a flow cytometer cells flow in a liquid stream one by one, where a lamps or laser beams in conjunction with sensors measure the intensity of reflected light from the cells. These measurements can be in linear or logarithmic space~\cite{practical-flow-cytometry-book}. The measured values depend on the light intensity projected onto cells which can be tuned by changing the voltage of the lasers or lamps. Different wavelengths correspond to different markers, but they might overlap. When the tail of the emission spectrum of a marker overlaps with the main part of the emission spectrum of another marker, it is called spillover as shown in Fig.~\ref{fig:flow-cytometry-spillover}~\cite{flow-cytometry-compensation}.

Compensating for spillover requires a spillover matrix ($SM$). $SP_{i,j}$ shows the percentage that marker $i$ spills over marker $j$. The compensation matrix ($CM$) is then the calculated as the inverse of the spill over matrix. Let $S$ be the true signal value, and $O$ be the observed value. Then we have~\footnote{\url{http://bioinformin.net/cytometry/compensation.php}}:

\begin{align}
  &CM = SM^{-1}\\
  &S = O \times CM
  \label{fml:fcs-compensation}
\end{align}

\begin{figure}[!ht]
  \centering
  \includegraphics[height=7cm]{figs/fcs-spillover}
  \caption{[DIRECT QUOTE] Fluorescence emission spectra for FITC and PE. The emission spectrum (the wavelengths of light generated by excitation of these molecules) is shown for an excitation at $488nm$ (the same as the argon-ion laser line). FITC emission is maximal at $\sim 515nm$; typically, a filter centered on $530nm$ is used to collect the emitted light (shaded region). The emission of is farther red, with a maximum at $\sim 575nm$; typically, a filter centered on this emission maximum is used to collect. Note that PE has some emission in the wavelength bands used to collect PE fluorescence (B); typically, the amount of light in the $575nm$ band is $\sim 15\%$ of that in the $530nm$ band (A). The PE has very little emission in the $530nm$ band (C), usually less than $2\%$ of the emission in the $575nm$ band (D)~\cite{flow-cytometry-compensation}.}
  \label{fig:flow-cytometry-spillover}
\end{figure}

\section{Data Preprocessing and Challenges}
Transformation and spillover compensation are the two main phases of raw flow cytometry data preprocessing.

\emph{Transformation}: The measured fluorescent intensities almost exponentially correspond to the number of existing fluorescent markers on or inside the cell. Therefor a proper transformation of the raw data is essential in order to have the data in a linear space. Logarithmic, log-linear hybrid transformation Logicle~\cite{fcs-logicle}, and hyperbolic arcsine~\cite{fcs-arcsineh} are some commonly used transformations. Some studies have compared different transformation techniques and reported their advantages and disadvantages~\cite{fcs-transformation-survey1, fcs-transformation-survey2}.

\emph{Spillover Compensation}: Compensation is done as shown in Formula~\ref{fml:fcs-compensation} and it relies on a given compensation or spillover matrix.

In practice data are produced through time and also maybe in different labs. This means reagent batches are different, and also flow cytometry machines are not necessarily calibrated alike, which also affects compensation matrices. Therefore normalization is a crucial step to make samples comparable~\cite{fcs-normalization}.

\section{High Dimensional Analysis and Visualization}
Manual analysis of flow cytometry data involves \emph{gating}. Researchers use density or scatter plots of one or two selected dimensions  of flow cytometry data in order to visualize and also select some areas on those plots to further investigate cells within the selected area. Visualization and further gating of those selected cells is commonly a next step to the analysis.

Manual gating of cells across several samples is a labor intensive and time consuming process. Not being able to analyze the data in its original higher dimensional space is another disadvantage of manual flow cytometry data analysis.


\subsection{flowType}

\subsection{RchyOptimyx}

\section{Lymphoma Diagnosis Quality Checking}
