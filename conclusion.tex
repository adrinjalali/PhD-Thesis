\chapter{Conclusion}
In chapter~\ref{sec:fcs} we covered a flow cytometry data analysis pipeline, and
its application to two types of non-Hodgkin lymphoma, \emph{i.e.} Follicular
Lymphoma and Diffuse Large B Cell Lymphomas. We showed how the pipeline can
extract novel insights from the data as well as effectively automate a laborious
workflow.

We showed how flowType can extract features which can be used to train a model
to classify subtypes. However, we did not consider the relationships between
those features, \emph{i.e.} cell populations, while training the model. One
approach would be to design a kernel which takes into account those
relationships and therefore implicitly reduce the dimensionality of the input
data.

In chapter~\ref{sec:adaptive-learning} we focused mostly on the analysis of DNA
methylation data using adaptive and interpretable models. Although we put a
heavy focus on the models, our experiments showed that the preprocessing step
can play a crucial role in stability and the performance of those models. In the
case of DNA methylation data, for instance, a step to aggregate methylation
levels over genes made the models more stable, faster, and better performing.
Our observations support the idea of putting more focus on the preprocessing
steps, and to document them in a more informative way. This would also greatly
help towards improved reproducibility of efforts in the field.
