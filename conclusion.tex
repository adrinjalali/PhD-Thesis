\chapter{Conclusion}
In chapter~\ref{sec:fcs} we covered a flow cytometry data analysis pipeline,
and its application to two types of non-Hodgkin lymphoma, \emph{i.e.}
Follicular Lymphoma and Diffuse Large B Cell Lymphomas. We showed how the
pipeline can extract novel insights from the data as well as effectively
automate a laborious workflow.

We showed how flowType can extract features which can be used to train a model
to classify subtypes. However, we did not consider the relationships between
those features, \emph{i.e.} cell populations, while training the model. One
approach would be to design a kernel which takes into account those
relationships and therefore implicitly reduce the dimensionality of the input
data.

In chapter~\ref{sec:adaptive-learning} we focused mostly on the analysis of DNA
methylation data using adaptive and interpretable models. Although we put a
heavy focus on the models, our experiments showed that the preprocessing step
can play a crucial role in stability and the performance of those models. In
the case of DNA methylation data, for instance, a step to aggregate methylation
levels over genes made the models more stable, faster, and better performing.
Our observations support the idea of putting more focus on the preprocessing
steps, and to document them in a more informative way. This would also greatly
help towards improved reproducibility of publications in the field.

This thesis was an effort towards better understanding of cancer as well as
improving the diagnosis process. However, there are a few aspects which need to
be done before these methods can best be employed in clinics:

\begin{itemize}
  \item Incorporating multiple data sources: the approaches we took in this
    thesis all incorporated a single input type at a time. Of course as
    mentioned in previous chapters, not all data sources are available for all
    patients or at the time of diagnosis. However, the same way that a
    pathologist would use different test results as accumulating evidence for a
    potential diagnosis or to support a treatment, a model should also be able
    to do the same and gain or loose confidence in a specific diagnosis as more
    data comes in for a single patient.
  \item Adaptive to missing data sources: incorporating increasing evidence
    also means that models should be able to cope with missing data and missing
    data sources. Our models in Chapter~\ref{sec:adaptive-learning} are efforts
    towards handling noisy or missing data in a single data source, but similar
    approaches can be taken to have an ensemble of models each working on a
    specific data source.
  \item Models confidences and reasoning/interpretability
  \item Real world issues such as batches, etc.
  \item Reproduce, deploy, test
  \item Counsel of doctors $\rightarrow$ counsel of models
\end{itemize}
